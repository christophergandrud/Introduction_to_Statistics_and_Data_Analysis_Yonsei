\documentclass{beamer}\usepackage{graphicx, color}
%% maxwidth is the original width if it is less than linewidth
%% otherwise use linewidth (to make sure the graphics do not exceed the margin)
\makeatletter
\def\maxwidth{ %
  \ifdim\Gin@nat@width>\linewidth
    \linewidth
  \else
    \Gin@nat@width
  \fi
}
\makeatother

\IfFileExists{upquote.sty}{\usepackage{upquote}}{}
\definecolor{fgcolor}{rgb}{0.2, 0.2, 0.2}
\newcommand{\hlnumber}[1]{\textcolor[rgb]{0,0,0}{#1}}%
\newcommand{\hlfunctioncall}[1]{\textcolor[rgb]{0.501960784313725,0,0.329411764705882}{\textbf{#1}}}%
\newcommand{\hlstring}[1]{\textcolor[rgb]{0.6,0.6,1}{#1}}%
\newcommand{\hlkeyword}[1]{\textcolor[rgb]{0,0,0}{\textbf{#1}}}%
\newcommand{\hlargument}[1]{\textcolor[rgb]{0.690196078431373,0.250980392156863,0.0196078431372549}{#1}}%
\newcommand{\hlcomment}[1]{\textcolor[rgb]{0.180392156862745,0.6,0.341176470588235}{#1}}%
\newcommand{\hlroxygencomment}[1]{\textcolor[rgb]{0.43921568627451,0.47843137254902,0.701960784313725}{#1}}%
\newcommand{\hlformalargs}[1]{\textcolor[rgb]{0.690196078431373,0.250980392156863,0.0196078431372549}{#1}}%
\newcommand{\hleqformalargs}[1]{\textcolor[rgb]{0.690196078431373,0.250980392156863,0.0196078431372549}{#1}}%
\newcommand{\hlassignement}[1]{\textcolor[rgb]{0,0,0}{\textbf{#1}}}%
\newcommand{\hlpackage}[1]{\textcolor[rgb]{0.588235294117647,0.709803921568627,0.145098039215686}{#1}}%
\newcommand{\hlslot}[1]{\textit{#1}}%
\newcommand{\hlsymbol}[1]{\textcolor[rgb]{0,0,0}{#1}}%
\newcommand{\hlprompt}[1]{\textcolor[rgb]{0.2,0.2,0.2}{#1}}%

\usepackage{framed}
\makeatletter
\newenvironment{kframe}{%
 \def\at@end@of@kframe{}%
 \ifinner\ifhmode%
  \def\at@end@of@kframe{\end{minipage}}%
  \begin{minipage}{\columnwidth}%
 \fi\fi%
 \def\FrameCommand##1{\hskip\@totalleftmargin \hskip-\fboxsep
 \colorbox{shadecolor}{##1}\hskip-\fboxsep
     % There is no \\@totalrightmargin, so:
     \hskip-\linewidth \hskip-\@totalleftmargin \hskip\columnwidth}%
 \MakeFramed {\advance\hsize-\width
   \@totalleftmargin\z@ \linewidth\hsize
   \@setminipage}}%
 {\par\unskip\endMakeFramed%
 \at@end@of@kframe}
\makeatother

\definecolor{shadecolor}{rgb}{.97, .97, .97}
\definecolor{messagecolor}{rgb}{0, 0, 0}
\definecolor{warningcolor}{rgb}{1, 0, 1}
\definecolor{errorcolor}{rgb}{1, 0, 0}
\newenvironment{knitrout}{}{} % an empty environment to be redefined in TeX

\usepackage{alltt}
\usetheme{Stats}
\setbeamercovered{transparent}
\usepackage{color}
\usepackage{hyperref}
  \hypersetup{
  	colorlinks=true
		linkcolor=black
		}
\usepackage{url}
\usepackage{graphics}
\usepackage{tikz}
\usepackage{booktabs}





%%%%%%%%%%%%%%%%%%%%%%%%%%%%%%%% Title Slide %%%%%%%%%%%%%%%%%%%%%%%%%%
\title[]{Intro to Social Science Data Analysis \\[1cm] Week 12 Lecture: Multivariate Linear Regression \& Presenting Regression Results}
\author[]{
    \href{mailto:gandrud@yonsei.ac.kr}{Christopher Gandrud}
}
\date{\today}


\begin{document}

\frame{\titlepage}

\section[Outline]{}
\frame{\tableofcontents}

%%%%%%%%%% Assignment 3
\section{Assignment 3}
\frame{
  \frametitle{Assignment 3}
  \begin{itemize}
    \item The Assignment 3 grades were extremely \textbf{bimodalally} distributed.
    \item We should care about \textbf{learning} rather than \textbf{grades}.
    \item So, if you received \textbf{below an 80\%} I want to \bf{work with you} to make sure you actually \textbf{learn} the material.
  \end{itemize}
}

\frame{
  \frametitle{Assignment 3}
  \begin{itemize}
    \item \textbf{By Friday} come to my office with:
    \begin{itemize}
      \item a \textbf{description} of what \textbf{steps} you need to take to complete each question,
      \item \textbf{identification} of where in the \textbf{lecture slides} you can find the code to complete these steps.
    \end{itemize}
    \item Then \textbf{redo} the assignment by next Tuesday (27 November) and you can receive \textbf{up to an 85\% on the assignment}.
  \end{itemize}
}

%%%%%%%%%% Assignment 4
\section{Assignment 4}
\frame{
  \frametitle{Assignment 4}
  Due: Friday 30 November \\[0.5cm]
  {\Large{Research Design}} \\[0.25cm]
  With your partner plan your research by answering the following questions:
  \begin{enumerate}
    \item What \textbf{difference} do you want to explain?
    \item What is your \textbf{best guess} explanation (i.e. thesis statement)?
    \item Can you \textbf{test your hypothesis using data}? If so, what data do you need to collect and what tests could you use?
    \item What \textbf{rival explanations} are their?
    \item How could you use data to test whether your best guess or the rival explanations are better? Write this as an \textbf{equation}.
  \end{enumerate}
{\tiny{Questionnaire from: modified from Cheryl Schonhardt-Bailey}}
}

%%%%%%%%%%% Recap
\section{Recap}
\frame{
	\frametitle{Quick Quiz 1}
{\Large{Interpret the following correlations ($R$):}} \\[0.5cm]
  \begin{itemize}
    \item<1-> $R = 0.91$
    \item<1-> $R = 0.02$
    \item<1-> $R = -0.3$
  \end{itemize}
}

\frame{
  \frametitle{Quick Quiz 2}
  What is a residual?\\[0.5cm]
  Why do we need to care about residuals in linear regression?
}

\frame{
  \frametitle{Quick Quiz 3}
  What assumptions does the linear regression model make?
}

\frame{
  \frametitle{Quick Quiz 4}
  Create a hypothesis test for a linear regression coefficient ($\hat{\beta}$)
}

\frame{
  \frametitle{Quick Quiz 5}
  How do you interpret a linear regression coefficient for a dummy variable?
}

%%%%%% Multiple Linear Regression
\section{Multiple Linear Regression}
\frame{
  \frametitle{Intro to Multiple Linear Regression}
  Last class we learned how to use the tools of simple linear regression to examine the \textbf{bivariate} relationship between a dependent variable and \textbf{one independent} variable. \\[0.5cm]
  What if we want to examine \textbf{multivariate relationships}, i.e. the relationship between a dependent variable and \textbf{multiple} independent variables at the same time? \\[0.5cm]
  In these cases we can use \textbf{multiple linear regression}.
}

\frame{
  \frametitle{Why?}
  \begin{center}
{\Large{\textbf{Why} would we want to examine multiple independent variables at the same time?}}
  \end{center}
}

\frame{
  \frametitle{Minimal criteria for making a causal argument}
  To make a \textbf{probabilistic causal argument}, i.e. ``$X$ caused $Y$" we need to meet \emph{at least} three criteria: \\[0.5cm]
  \begin{itemize}
    \item $X$ is \textbf{statistically associated} with $Y$,
    \item $X$ happens before $Y$ (i.e. \textbf{time order}),
    \item all \textbf{alternative explanations} for the association are ruled out.
  \end{itemize}    
}

\frame{
  \frametitle{Time Order \& Causality}
  {\Large{Linear regression per se can't help us establish time order. \\[0.5cm]
  We need to understand our data to do that. \\[0.5cm]
  We may also need to take measurements at multiple points in time and use more advanced statistical tools than the ones covered in this course.}}
}

\frame{
  \frametitle{Simple Linear Regression \& Causality}
{\Large{Simple Linear Regression is a tool we can use to establish the statistical association between $X$ and $Y$. \\[0.5cm]
  How can we determine how much if at all, the value of $Y$ is actually explained by the value of $X$ and not some other factor(s)?}}
}

\frame{
  \frametitle{Spurious Example}
{\Large{A silly example:}} \\[0.5cm]
  You are interested in what causes expensive fire damage. \\[0.25cm]
  You observe that the most expensive fires have the most fire trucks on the scene. \\[0.25cm]
  Did having more fire trucks cause more damage?
}

\frame{
  \frametitle{Spurious Example}
  Clearly, the \emph{size of the fire} caused \textbf{both} the amount of fire damage and the number of fire trucks that responded. \\[0.5cm]
  There is a \textbf{spurious relationship} between number of fire trucks and fire damage.
}

\frame{
  \frametitle{Spurious Diagram}
  \begin{figure}
    \caption{Spurious Relationship}
    \begin{tikzpicture}
      \vspace{0.5cm}
      %% Nodes
      \node (z) at (0,0) {$Z$};
      \node (x) at (-2, 2) {$X$};
      \node (y) at (2, 2) {$Y$};
      
      %% Arrows
      \draw[->] (z) -- (x);
      \draw[->] (z) -- (y);
      
    \end{tikzpicture}
  \end{figure}
}

\frame{
  \frametitle{Controlling for $Z$}
  If we were able to run an experiment where we \textbf{randomized} the units who are given `treatment' $X$ and those that are not (the `control' group), \\[0.5cm]
  \textbf{on average} the units will have the same values of $Z$. \\[0.5cm]
  We can say that we are \bf{controlling for} $Z$. \\[0.5cm]
  If, after randomization, the association between $X$ and $Y$ still exists, then we have found evidence to rule out alternative explanations.
}

\frame{
  \frametitle{Observational Data}
  However, in many social science situations we cannot run an experiment with randomized control and treatment groups. \\[0.5cm]
  For example, we cannot randomly assign people to live in dictatorships and democracies. \\[0.5cm]
  In these cases we need to use \textbf{statistical control} like \bf{multiple linear regression}.
}

\frame{
  \frametitle{Note:}
  In many cases social scientists actually do conduct randomized experiments. \\[0.5cm]
  For example, the Obama campaign randomized the email messages it sent to people asking for donations. \\[0.5cm]
  Also, there are more advanced statistical techniques that can be combined with multiple linear regression to enhance statistical control. For example, matching.
}

\frame{
  \frametitle{Multi-causality}
{\LARGE{Multi-causality:}}\\[0.5cm]
  In the social sciences something \emph{rarely has one cause}. \\[0.5cm]
  Instead, phenomenon usually have multiple causes; each making a contribution to the probable value of an outcome. \\[0.5cm]
  Multiple linear regression is a statistical tool that we can use to help identify the \textbf{individual} contribution of some factor to an outcome, \textbf{controlling for} other factors.
}

\frame{
  \frametitle{The Multiple Linear Regression Model}
  An estimated multiple linear regression model for predictors $x_{1} \ldots x_{p}$:
  \[
  \hat{y} = \alpha + \beta_{1}x_{1} + \beta_{2}x_{2} + \ldots + \beta_{p}x_{p} 
  \]
}

\begin{frame}[fragile]
  \frametitle{Today's Data}
\begin{knitrout}
\definecolor{shadecolor}{rgb}{0.969, 0.969, 0.969}\color{fgcolor}\begin{kframe}
\begin{alltt}
\hlcomment{# Load library}
\hlfunctioncall{library}(openintro)

\hlcomment{# Load data}
\hlfunctioncall{data}(marioKart)

\hlcomment{# Show Variables}
\hlfunctioncall{names}(marioKart)
\end{alltt}
\begin{verbatim}
##  [1] "ID"         "duration"   "nBids"     
##  [4] "cond"       "startPr"    "shipPr"    
##  [7] "totalPr"    "shipSp"     "sellerRate"
## [10] "stockPhoto" "wheels"     "title"
\end{verbatim}
\end{kframe}
\end{knitrout}

\end{frame}

\frame{
  \frametitle{Example Multiple Linear Regression Model}
  Imagine we are interested in what explains the EBay selling price of the game Mario Kart (\texttt{totalPr})? \\[0.5cm]
  We want to see if the duration of the auction in days (\texttt{duration}), whether the game was in used (\texttt{condused}), and the number of wheels included in the auction (\texttt{wheels}) impacted the selling price.
}

\begin{frame}[fragile]
  \frametitle{Scatter}
\begin{knitrout}
\definecolor{shadecolor}{rgb}{0.969, 0.969, 0.969}\color{fgcolor}

{\centering \includegraphics[width=\maxwidth]{figure/Scatter} 

}


\end{knitrout}

\end{frame}

\frame{
  \frametitle{Sum of Squared Residuals}
  Estimating the coefficients and making inferences about them is similar to simple linear regression. \\[0.5cm]
  For example, to find the sum of the squared residuals:
  \[
  SSR = \sum^{n}_{i=1} e^{2}_{i} = \sum^{n}_{i=1} (y_{i} + \hat{y}_{i})^{2} 
  \]
}

\begin{frame}[fragile]
  \frametitle{Estimation}
  Estimating the $\beta$s by hand in multiple linear regression is very difficult, but it is relatively easy if you let R do the hard work with the \texttt{lm} command:
\begin{knitrout}
\definecolor{shadecolor}{rgb}{0.969, 0.969, 0.969}\color{fgcolor}\begin{kframe}
\begin{alltt}
\hlcomment{# Estimated multivariate linear regression model}
M1 <- \hlfunctioncall{lm}(totalPr ~ duration + cond + wheels, 
         data = marioKart)
\end{alltt}
\end{kframe}
\end{knitrout}

\end{frame}

\begin{frame}[fragile]
  \frametitle{Showing the Coefficient Estimates}
{\scriptsize{
\begin{knitrout}
\definecolor{shadecolor}{rgb}{0.969, 0.969, 0.969}\color{fgcolor}\begin{kframe}
\begin{alltt}
\hlcomment{# Show coefficient point estimates}
M1
\end{alltt}
\begin{verbatim}
## 
## Call:
## lm(formula = totalPr ~ duration + cond + wheels, data = marioKart)
## 
## Coefficients:
## (Intercept)     duration     condused  
##      35.735        0.680       -0.695  
##      wheels  
##      10.455
\end{verbatim}
\end{kframe}
\end{knitrout}

}}
What is the estimated linear regression equation?
\end{frame}

\frame{
  \frametitle{Linear Regression Equation}
{\small{
  \[
  \widehat{\tt{totalPr}} = 35.74 + 0.68(\tt{duration}) + -0.695(\tt{condused}) + 10.46(\tt{wheels})
  \]
}}
}

\frame{
  \frametitle{Question Linear Regression Equation}
{\small{
  \[
  \widehat{\tt{totalPr}} = 35.74 + 0.68(\tt{duration}) + -0.695(\tt{condused}) + 10.46(\tt{wheels})
  \]
}} \\[0.5cm]
  What do you estimated will be the total selling price for a Mario Kart auction that was 5 days long, was in new condition, and included 2 wheels?
}

\frame{
  \frametitle{Linear Regression Equation Estimated $Y$}
  \[
  60.06 = 35.74 + 0.68(5) + -0.695(0) + 10.46(2)
  \]
}

\frame{
  \frametitle{Single Variable Interpretation}
  We can interpret the effect of \texttt{duration} like this:
  \begin{quote}
    Controlling for the condition of the game and the number of wheels included, each day a Mario Kart EBay auction continues I expect that the total selling price will increase by 0.68.
  \end{quote}
   \hline \vspace{0.5cm}
For the dummy variable \texttt{condused} we have the following interpretation:
\begin{quote}
    Controlling for the duration of the auction and the number of wheels sold, I expect used Mario Kart games to sell for 0.695 less than new games in EBay auctions.
 \end{quote}
}

\frame{
  \frametitle{Remember the Units}
  \begin{center}
{\Large{Remember that the coefficients are in terms of the variables' units. \\[0.5cm]
  A large coefficient does not necessarily mean a big effect.}}
  \end{center}
}

%%% Multinomial variables
\begin{frame}[fragile]
  \frametitle{Multinomial Variables in Multiple Linear Regression}
  \begin{center}
  What if an independent variable has more than two category?
  \end{center}
  For example,  the shipping method variable \texttt{shipSp} has the following categories:
\begin{knitrout}
\definecolor{shadecolor}{rgb}{0.969, 0.969, 0.969}\color{fgcolor}\begin{kframe}
\begin{alltt}
\hlfunctioncall{summary}(marioKart$shipSp)
\end{alltt}
\begin{verbatim}
## firstClass      media      other     parcel 
##         22         14          3         16 
##   priority   standard    ups3Day  upsGround 
##         23         33          1         31
\end{verbatim}
\end{kframe}
\end{knitrout}

\end{frame}

\begin{frame}[fragile]
  \frametitle{Linear Model with Multinomial Independent Variables}
\begin{knitrout}
\definecolor{shadecolor}{rgb}{0.969, 0.969, 0.969}\color{fgcolor}\begin{kframe}
\begin{alltt}
\hlcomment{# Estimate model}
M2 <- \hlfunctioncall{lm}(totalPr ~ shipSp, data = marioKart)
\end{alltt}
\end{kframe}
\end{knitrout}


{\scriptsize{
\begin{knitrout}
\definecolor{shadecolor}{rgb}{0.969, 0.969, 0.969}\color{fgcolor}\begin{kframe}
\begin{alltt}
M2
\end{alltt}
\begin{verbatim}
## 
## Call:
## lm(formula = totalPr ~ shipSp, data = marioKart)
## 
## Coefficients:
##     (Intercept)      shipSpmedia  
##           42.53             8.50  
##     shipSpother     shipSpparcel  
##            4.46            26.84  
##  shipSppriority   shipSpstandard  
##            1.45             4.03  
##   shipSpups3Day  shipSpupsGround  
##            4.47            10.27
\end{verbatim}
\end{kframe}
\end{knitrout}

}}
\end{frame}

\begin{frame}[fragile,plain]
{\footnotesize{
\begin{knitrout}
\definecolor{shadecolor}{rgb}{0.969, 0.969, 0.969}\color{fgcolor}\begin{kframe}
\begin{alltt}
\hlcomment{# Show results}
M2
\end{alltt}
\begin{verbatim}
## 
## Call:
## lm(formula = totalPr ~ shipSp, data = marioKart)
## 
## Coefficients:
##     (Intercept)      shipSpmedia  
##           42.53             8.50  
##     shipSpother     shipSpparcel  
##            4.46            26.84  
##  shipSppriority   shipSpstandard  
##            1.45             4.03  
##   shipSpups3Day  shipSpupsGround  
##            4.47            10.27
\end{verbatim}
\end{kframe}
\end{knitrout}

}}
\end{frame}

\frame{
  \frametitle{The Linear Regression Equation with a Multinomial Independent Variable}
  \[
  \begin{split}
  \widehat{totalPr} = & \left 42.35 + 8.5(\tt{Media}) + 4.46(\tt{Other}) + \right. \\
                      & \left 26.84(\tt{Parcel}) + 1.45(\tt{Priority}) + \right. \\
                      & \left 4.03(\tt{Standard}) + 4.47(\tt{ups3Day}) + \right.\\
                      & \left 10.27(\tt{upsGround}) \right
  \end{split}
  \]
  Note that \texttt{firstclass} is missing. \\[0.5cm]
  It is the \textbf{reference category}. 
}

\frame{
  \frametitle{Interpreting Multinomial Variable Regression Coefficients}
  The coefficients \textbf{compare} the predicted effect of the category to the \textbf{reference category}. \\[0.5cm]
  For example, the coefficient for \texttt{Parcel} post is 26.84. This means that:
  \begin{quotation}
    We estimate that Mario Kart games that are shipped by parcel post will sell for 26.84 more than those sold with first class shipping.
  \end{quotation}\\[0.5cm]
  Question, in this example, what is the predicted selling price of a game sold with first class shipping?
}

\begin{frame}[fragile]
  \frametitle{Warning}
  There are at least two issues you should look out for with multinomial variables in linear regression: \\[0.5cm]
  \begin{itemize}
    \item Is the comparison with the reference category substantively interesting? If not \textbf{change the reference category} with the \texttt{relevel} command.
    \item There may be too many categories with too few observations per category to estimate meaningful results. If this is the case, \textbf{combine some of the categories in a substantively meaningful way}.
  \end{itemize}
\end{frame}

%%% Hypothesis testing
\section{Hypothesis Testing with Multiple Linear Regression}
\frame{
  \frametitle{Hypothesis Testing and Multiple Linear Regression}
  \begin{center}
{\LARGE{Hypothesis Testing and Multiple Linear Regression}}
  \end{center}
}

\frame{
  \frametitle{Hypothesis Testing with Multiple Linear Regression}
  We follow the same steps as before to test the null hypothesis that:
  \[
  \beta_{p} = 0,\:\mathrm{when\;the\;other\;variables\;are\;included}
  \]
  With the alternative hypothesis that:
  \[
  \beta_{p} \neq 0,\:\mathrm{when\;the\;other\;variables\;are\;included}
  \]
}

\frame{
  \frametitle{Hypothesis Testing with Multiple Linear Regression}
  Once we have our hypotheses: \\[0.5cm]
{\LARGE{First:}} \\[0.5cm]
  calculated the test statistic ($t\:test$). \\[0.5cm]
{\LARGE{Second:}} \\[0.5cm]
  find the p-value or confidence interval for the significance level we are interested in.
}

\begin{frame}[fragile]
  \frametitle{P-values \& Summary Table}
  Luckily, R will quickly do this for us with the \texttt{summary} command:
{\tiny{
\begin{knitrout}
\definecolor{shadecolor}{rgb}{0.969, 0.969, 0.969}\color{fgcolor}\begin{kframe}
\begin{alltt}
\hlfunctioncall{summary}(M1)
\end{alltt}
\begin{verbatim}
## 
## Call:
## lm(formula = totalPr ~ duration + cond + wheels, data = marioKart)
## 
## Residuals:
##    Min     1Q Median     3Q    Max 
## -17.25  -6.26  -2.73   0.51 265.80 
## 
## Coefficients:
##             Estimate Std. Error t value Pr(>|t|)
## (Intercept)   35.735      5.889    6.07  1.2e-08
## duration       0.680      0.912    0.75  0.45724
## condused      -0.695      5.036   -0.14  0.89051
## wheels        10.455      2.697    3.88  0.00016
##                
## (Intercept) ***
## duration       
## condused       
## wheels      ***
## ---
## Signif. codes:  0 '***' 0.001 '**' 0.01 '*' 0.05 '.' 0.1 ' ' 1 
## 
## Residual standard error: 24.5 on 139 degrees of freedom
## Multiple R-squared: 0.113,	Adjusted R-squared: 0.0935 
## F-statistic: 5.88 on 3 and 139 DF,  p-value: 0.000825
\end{verbatim}
\end{kframe}
\end{knitrout}

}}
\end{frame}

\begin{frame}[fragile,plain]

\begin{table}[!ht]
\caption{Linear Regression for Mario Kart Total Selling Price}
\label{} 
\begin{tabular}{ l D{.}{.}{2} } 
\hline 
  & \multicolumn{ 1 }{ c }{ Model 1 } \\ \hline
 %            & Model 1     \\ 
(Intercept)  & 35.73 ^{***}\\ 
             & (5.89)      \\ 
duration     & 0.68        \\ 
             & (0.91)      \\ 
condused     & -0.69       \\ 
             & (5.04)      \\ 
wheels       & 10.45 ^{***}\\ 
             & (2.70)       \\
 $N$          & 143         \\ 
$R^2$        & 0.11        \\ 
adj. $R^2$   & 0.09        \\ 
Resid. sd    & 24.46        \\ \hline
 \multicolumn{2}{l}{\footnotesize{Standard errors in parentheses}}\\
\multicolumn{2}{l}{\footnotesize{$^\dagger$ significant at $p<.10$; $^* p<.05$; $^{**} p<.01$; $^{***} p<.001$}} 
\end{tabular} 
 \end{table}



\end{frame}

%%% Confidence Intervals
\frame{
  \frametitle{Confidence Intervals for Multiple Linear Regression Coefficients}
  Estimating confidence intervals for multiple linear regression coefficients is similar to what we have done before:
  \[
  \beta_{i} \pm t^{*}_{df}SE_{\beta_{i}}
  \]
  where $t^{*}_{df}$ is the $t\: test$ statistic with $df = n - p - 1$.
}

\begin{frame}[fragile]
  \frametitle{Confidence Intervals in R for Objects of Class \texttt{lm}}
  To calculate the confidence intervals for regression coefficients in R use the \texttt{confint} command.
\begin{knitrout}
\definecolor{shadecolor}{rgb}{0.969, 0.969, 0.969}\color{fgcolor}\begin{kframe}
\begin{alltt}
\hlcomment{# Find confidence intervals}
CI <- \hlfunctioncall{confint}(M1)
CI
\end{alltt}
\begin{verbatim}
##               2.5 % 97.5 %
## (Intercept)  24.091 47.379
## duration     -1.124  2.484
## condused    -10.652  9.263
## wheels        5.122 15.788
\end{verbatim}
\end{kframe}
\end{knitrout}

\end{frame}

\begin{frame}[fragile]
  \frametitle{Graphically}
\begin{knitrout}
\definecolor{shadecolor}{rgb}{0.969, 0.969, 0.969}\color{fgcolor}

{\centering \includegraphics[width=\maxwidth]{figure/ConfIntGraph} 

}


\end{knitrout}

\end{frame}

%%%%% Model Assumptions
\section{Model Assumptions}
\frame{
  \frametitle{Multiple Linear Regression Model Assumptions} 
  {\Large{Multiple Linear Regression Model Assumptions}} \\[0.5cm]
  \begin{itemize}
    \item Nearly \textbf{normally distributed residuals},
    \item Nearly \textbf{constant residual variability},
    \item Residuals are \textbf{independent},
    \item Each variable is \textbf{linearly} associated with the outcome
  \end{itemize}
}

%%% Model Selection (R2)
\section{Model Selection}
\frame{
  \frametitle{Model Selection with the Adjusted R Squared}
  \begin{center}
{\Large{How do we decide which variables to include in our multiple linear regression model?}}
  \end{center}
}

\frame{
  \frametitle{Omitted Variable Bias}
  We generally want to include all of the variables that have an important effect on the outcome. \\[0.5cm]
  Not including them creates \textbf{omitted variable bias}: i.e. our results are biased because we have omitted important variables.
}

\frame{
  \frametitle{However\ldots}
  We don't want to just include every variable we can think of into one model. \\[0.5cm]
  \begin{itemize}
    \item \textbf{Occam's Razor}: aim for parsimony (the simplest model possible).
    \item The more variables you add, the {\bf{fewer degrees of freedom}} you will have to work with.
    \item You may include \textbf{highly correlated variables}, which can produce very biased estimates.
  \end{itemize}
}

\frame{
  \frametitle{The Adjusted $R^{2}$}
  We can use the \textbf{adjusted $R^{2}$} ($R^{2}_{adj}$) to determine if the addition of a variable to a linear regression model \textbf{reduces the errors} we make when predicting the outcome. 
  \[
  R^{2}_{adj} = \frac{Var(e_{i})/n - p - 1}{Var(y_{i})/(n - 1)}
  \]
  It is similar to the non-adjusted $R^{2}$ we learned about last week, but \textbf{only gets bigger if the addition of a variable reduces the errors} we make predicting the outcome.
}

\begin{frame}[fragile,plain]

\begin{table}[!ht]
\caption{Linear Regressions for Mario Kart Total Selling Price}
\label{} 
\begin{tabular}{ l D{.}{.}{2}D{.}{.}{2}D{.}{.}{2} } 
\hline 
  & \multicolumn{ 1 }{ c }{ Model 1 } & \multicolumn{ 1 }{ c }{ Model 2 } & \multicolumn{ 1 }{ c }{ Model 3 } \\ \hline
 %            & Model 1      & Model 2      & Model 3     \\ 
(Intercept)  & 38.41 ^{***} & 35.38 ^{***} & 35.73 ^{***}\\ 
             & (3.43)       & (5.27)       & (5.89)      \\ 
wheels       & 10.01 ^{***} & 10.58 ^{***} & 10.45 ^{***}\\ 
             & (2.41)       & (2.53)       & (2.70)      \\ 
duration     &              & 0.63         & 0.68        \\ 
             &              & (0.83)       & (0.91)      \\ 
condused     &              &              & -0.69       \\ 
             &              &              & (5.04)       \\
 $N$          & 143          & 143          & 143         \\ 
$R^2$        & 0.11         & 0.11         & 0.11        \\ 
adj. $R^2$   & 0.10         & 0.10         & 0.09        \\ 
Resid. sd    & 24.34        & 24.37        & 24.46        \\ \hline
 \multicolumn{4}{l}{\footnotesize{Standard errors in parentheses}}\\
\multicolumn{4}{l}{\footnotesize{$^\dagger$ significant at $p<.10$; $^* p<.05$; $^{**} p<.01$; $^{***} p<.001$}} 
\end{tabular} 
 \end{table}



\end{frame}

\frame{
  \frametitle{Model Section as Art}
  Model selection is something of an art, but it is useful to follow these \textbf{rules of thumb}: \\[0.5cm]
  \begin{itemize}
    \item \textbf{Drop highly insignificant variables}, especially if they don't add to the adjusted $R^{2}$
    \item \textbf{Show a number of step-wise models} that could convince the reader that you have found the most {\bf{parsimonious}} model.
    \item Include results from \textbf{theoretically important} variables,
    \item \textbf{Avoid including highly correlated} variables in the same model,
    \item Don't show estimates from models that \textbf{violate linear regression assumptions},
  \end{itemize}
}

%%% Simulating expected values
\section{Simulating Expected Values}
\begin{frame}[fragile,plain]
  Though popular, tables like these are \textbf{not an effective way to communicate your findings}. \\[0.2cm]
  \begin{center}

 
\begin{tabular}{ l D{.}{.}{2}D{.}{.}{2}D{.}{.}{2} } 
\hline 
  & \multicolumn{ 1 }{ c }{ Model 1 } & \multicolumn{ 1 }{ c }{ Model 2 } & \multicolumn{ 1 }{ c }{ Model 3 } \\ \hline
 %            & Model 1      & Model 2      & Model 3     \\ 
(Intercept)  & 38.41 ^{***} & 35.38 ^{***} & 35.73 ^{***}\\ 
             & (3.43)       & (5.27)       & (5.89)      \\ 
wheels       & 10.01 ^{***} & 10.58 ^{***} & 10.45 ^{***}\\ 
             & (2.41)       & (2.53)       & (2.70)      \\ 
duration     &              & 0.63         & 0.68        \\ 
             &              & (0.83)       & (0.91)      \\ 
condused     &              &              & -0.69       \\ 
             &              &              & (5.04)       \\
 $N$          & 143          & 143          & 143         \\ 
$R^2$        & 0.11         & 0.11         & 0.11        \\ 
adj. $R^2$   & 0.10         & 0.10         & 0.09        \\ 
Resid. sd    & 24.34        & 24.37        & 24.46        \\ \hline
 \multicolumn{4}{l}{\footnotesize{Standard errors in parentheses}}\\
\multicolumn{4}{l}{\footnotesize{$^\dagger$ significant at $p<.10$; $^* p<.05$; $^{**} p<.01$; $^{***} p<.001$}} 
\end{tabular} 



  \end{center}
\end{frame}


\frame{
  \frametitle{Simulating Expected Values}
  Instead, King et al (2000) argue that we should find a way to \\[0.5cm]
  \begin{itemize} 
    \item \textbf{present results}, 
    \item including our \textbf{estimation uncertainty} 
    \item in relatively \textbf{simple language}.
    \end{itemize}
}

\frame{
  \frametitle{Their Suggestion}
  They suggest that we simulate probable outcomes from our estimated models. \\[0.5cm]
  They (and others) created the \texttt{Zelig} package to help us simulate and graph expected outcomes.
}

\frame{
  \frametitle{\texttt{Zelig} Simulation Steps}
{\Large{\texttt{Zelig} Simulation Steps:}} \\[0.5cm]
  \begin{enumerate}
    \item Estimate model,
    \item Set fitted values,
    \item Simulate expected outcomes,
    \item Graph the simulated values.
  \end{enumerate}
}

\begin{frame}[fragile]
  \frametitle{\texttt{Zelig} Example}
  Simulated expected Total Mario Kart auction price with various numbers of included steering wheels.
\begin{knitrout}
\definecolor{shadecolor}{rgb}{0.969, 0.969, 0.969}\color{fgcolor}\begin{kframe}
\begin{alltt}
\hlcomment{# Load Zelig}
\hlfunctioncall{library}(Zelig)

\hlcomment{# Estimate model of total auction price with}
\hlcomment{# wheels and duration as independent variables}
ZOut <- \hlfunctioncall{zelig}(totalPr ~ wheels + duration,
              data = marioKart, model = \hlstring{"normal"},
              cite = FALSE)
\end{alltt}
\end{kframe}
\end{knitrout}

\end{frame}

\begin{frame}[fragile]
\begin{knitrout}
\definecolor{shadecolor}{rgb}{0.969, 0.969, 0.969}\color{fgcolor}\begin{kframe}
\begin{alltt}
\hlcomment{# Find valid range of wheels values}
\hlfunctioncall{range}(marioKart$wheels)
\end{alltt}
\begin{verbatim}
## [1] 0 4
\end{verbatim}
\begin{alltt}
\hlcomment{# Set fitted values for wheels}
\hlcomment{# Note: duration set at its mean}
WValues <- 1:4
XOut <- \hlfunctioncall{setx}(ZOut, wheels = WValues)

\hlcomment{# Simulate expected prices}
ZSim <- \hlfunctioncall{sim}(ZOut, x = XOut)
\end{alltt}
\end{kframe}
\end{knitrout}

\end{frame}

\begin{frame}[fragile]
\begin{knitrout}
\definecolor{shadecolor}{rgb}{0.969, 0.969, 0.969}\color{fgcolor}\begin{kframe}
\begin{alltt}
\hlcomment{# Plot middle 95% of the simulations}
\hlfunctioncall{plot}(ZSim, ylab = \hlstring{"Expected Price"}, 
     main = \hlstring{"Effect of Wheels on MK EBay Auctions"})
\end{alltt}
\end{kframe}

{\centering \includegraphics[width=\maxwidth]{figure/Zelig3} 

}


\end{knitrout}

\end{frame}

\frame{
  \frametitle{Question}
  \begin{center}
{\Large{Why is the predicted range of prices wider when there are 4 steering wheels?}}
  \end{center}
}

\frame{
  \frametitle{With (a lot) more work simulation graphs can look better}
  \begin{center}
    \includegraphics[scale=0.65]{/git_repositories/GreenBook/Paper/figure/ExpectValueParty.pdf}
  \end{center}
}
%%% References
\begin{frame}[allowframebreaks]
  \frametitle{References}
  Crawley, Michael J. 2005. Statistics: An Introduction Using R. Chichester: John Wiley \& Sons. Ltd. \\[0.25cm]
  Diaz, David M., Christopher D. Barr, and Mine \c{C}etinkaya-Rundel. 2011. OpenIntro 
  Statistics. 1st ed. \url{http://www.openintro.org/stat/downloads.php}. \\[0.25cm] 
  King, Gary, Michael Tomz, and Jason Wittenberg. 2000. “Making the Most of Statistical Analyses: Improving Interpretation and Presentation.” American Journal of Political Science 44(2): 347–361.
\end{frame}

\end{document}
