\documentclass{beamer}\usepackage{graphicx, color}
%% maxwidth is the original width if it is less than linewidth
%% otherwise use linewidth (to make sure the graphics do not exceed the margin)
\makeatletter
\def\maxwidth{ %
  \ifdim\Gin@nat@width>\linewidth
    \linewidth
  \else
    \Gin@nat@width
  \fi
}
\makeatother

\IfFileExists{upquote.sty}{\usepackage{upquote}}{}
\definecolor{fgcolor}{rgb}{0.2, 0.2, 0.2}
\newcommand{\hlnumber}[1]{\textcolor[rgb]{0,0,0}{#1}}%
\newcommand{\hlfunctioncall}[1]{\textcolor[rgb]{0.501960784313725,0,0.329411764705882}{\textbf{#1}}}%
\newcommand{\hlstring}[1]{\textcolor[rgb]{0.6,0.6,1}{#1}}%
\newcommand{\hlkeyword}[1]{\textcolor[rgb]{0,0,0}{\textbf{#1}}}%
\newcommand{\hlargument}[1]{\textcolor[rgb]{0.690196078431373,0.250980392156863,0.0196078431372549}{#1}}%
\newcommand{\hlcomment}[1]{\textcolor[rgb]{0.180392156862745,0.6,0.341176470588235}{#1}}%
\newcommand{\hlroxygencomment}[1]{\textcolor[rgb]{0.43921568627451,0.47843137254902,0.701960784313725}{#1}}%
\newcommand{\hlformalargs}[1]{\textcolor[rgb]{0.690196078431373,0.250980392156863,0.0196078431372549}{#1}}%
\newcommand{\hleqformalargs}[1]{\textcolor[rgb]{0.690196078431373,0.250980392156863,0.0196078431372549}{#1}}%
\newcommand{\hlassignement}[1]{\textcolor[rgb]{0,0,0}{\textbf{#1}}}%
\newcommand{\hlpackage}[1]{\textcolor[rgb]{0.588235294117647,0.709803921568627,0.145098039215686}{#1}}%
\newcommand{\hlslot}[1]{\textit{#1}}%
\newcommand{\hlsymbol}[1]{\textcolor[rgb]{0,0,0}{#1}}%
\newcommand{\hlprompt}[1]{\textcolor[rgb]{0.2,0.2,0.2}{#1}}%

\usepackage{framed}
\makeatletter
\newenvironment{kframe}{%
 \def\at@end@of@kframe{}%
 \ifinner\ifhmode%
  \def\at@end@of@kframe{\end{minipage}}%
  \begin{minipage}{\columnwidth}%
 \fi\fi%
 \def\FrameCommand##1{\hskip\@totalleftmargin \hskip-\fboxsep
 \colorbox{shadecolor}{##1}\hskip-\fboxsep
     % There is no \\@totalrightmargin, so:
     \hskip-\linewidth \hskip-\@totalleftmargin \hskip\columnwidth}%
 \MakeFramed {\advance\hsize-\width
   \@totalleftmargin\z@ \linewidth\hsize
   \@setminipage}}%
 {\par\unskip\endMakeFramed%
 \at@end@of@kframe}
\makeatother

\definecolor{shadecolor}{rgb}{.97, .97, .97}
\definecolor{messagecolor}{rgb}{0, 0, 0}
\definecolor{warningcolor}{rgb}{1, 0, 1}
\definecolor{errorcolor}{rgb}{1, 0, 0}
\newenvironment{knitrout}{}{} % an empty environment to be redefined in TeX

\usepackage{alltt}
\usetheme{Stats}
\setbeamercovered{transparent}
\usepackage{color}
\usepackage{hyperref}
  \hypersetup{
  	colorlinks=true
		linkcolor=black
		}
\usepackage{url}
\usepackage{graphics}
\usepackage{tikz}
\usepackage{booktabs}


%%%%%%%%%%%%%%%%%%%%%%%%%%%%%%%% Title Slide %%%%%%%%%%%%%%%%%%%%%%%%%%
\title[Seminar 1]{Intro to Social Science Data Analysis \\[1cm] Seminar 1: Introduction to R and RStudio \\[0.25cm]}
\author[]{
    \href{mailto:gandrud@yonsei.ac.kr}{Christopher Gandrud}
}
\date{\today}


\begin{document}

\frame{\titlepage}

\section[Outline]{}
\frame{\tableofcontents}

\section{What is the seminar for?}
\frame{
  \frametitle{Seminar Purpose (1)}
  \begin{itemize}
    \item<1->  This course is about learning skills that will help you {\bf{gather}}, {\bf{analyse}}, and {\bf{present}} social science data.
    \item<2-> The best way to develop these skills is by {\bf{using}} them.
    \item<3-> The seminar is an opportunity for you to {\bf{practice}} using these tools. Here you can:
      \begin{itemize}
        \item<4-> Ask me questions,
        \item<5-> Collaborate with your fellow students.
      \end{itemize}
  \end{itemize}
}

\frame{
  \frametitle{Format (1)}
    \begin{itemize}
      \item<1-> In the lecture \& seminar I will give you {\bf{general tools}}.
      \item<2-> In the seminar I will give you a {\bf{goal}} to complete with these tools (and others).
    \end{itemize}
}

\frame{
  \frametitle{Format (2)}
    {\LARGE{Note: There is {\bf{rarely only one correct answer}}.}} \\[0.5cm]
    
    {\LARGE{I want you to {\bf{creatively}} use the tools and resources available to you.}} \\[0.5cm]
    
    {\LARGE{I do not want you to just copy a list of instructions.}}
}

\section{Getting Started with RStudio}
\frame{
  \frametitle{Getting Started with RStudio}
    \begin{center}
      {\LARGE{Open Rstudio}} \\[0.5cm]
      \includegraphics{images/RStudioIcon.png}
    \end{center}
}

\frame{
  \frametitle{Looking Around}
  {\LARGE{Look around the main Panel.}} \\[0.5cm]
    \begin{itemize}
      \item {\bf{Console}}: Where you can enter R code.
      \item {\bf{Workspace/History}}: Where you can see your objects and the history of commands.
      \item {\bf{Files/Plots/Packages/Help}}: Navigate files, see the graphs you make and your packages, read help files.
    \end{itemize}
}

\frame{
  \frametitle{Source Files}
  {\LARGE{Create a new {\bf{source code}} file:}} \\[0.5cm]
    \begin{itemize}
      \item<1-> Click: {\tt{File}} \rightarrow {\tt{New}} \rightarrow {\tt{R\: Script}}
      \item<2-> Usually write your R code here and {\bf{save your source files} to Dropbox}.
      \item<3-> This will make your life a {\bf{lot easier}}.
    \end{itemize}
}

\frame{
  \frametitle{Source Files}
  {\LARGE{Create a new {\bf{notebook}}:}} \\[0.5cm]
    \begin{itemize}
      \item<1-> Notebooks allow you to record {\bf{what}} you do and {\bf{how}} you do it.
      \item<2-> When you have you source code file open, click: {\tt{File}} \rightarrow {\tt{Compile\: Notebook \ldots}} 
      \item<3-> Compile a notebook when you are finished.
      \item<4-> We will do more of this in Week 4.
    \end{itemize}
}

\begin{frame}[fragile]
  \frametitle{Commenting}
  {\bf{Hint:}} You can make your code easier to read by {\bf{regularly commenting}} on it. \\[0.5cm]
      Use the {\tt{\#}} (hash). For example,
\begin{kframe}
\begin{alltt}
\hlcomment{# This is a comment}
\end{alltt}
\end{kframe}

\end{frame}

\section{Getting Started with R}
\frame{
  \frametitle{The Basics: Objects (1)}
  {\LARGE{Objects}} \\[0.5cm]
  \begin{itemize}
    \item<1-> R is a computer {\bf{language}}, mostly used for statistical analysis.
      \begin{itemize}
        \item<1-> The rules for writing the R language is called its {\bf{syntax}}.
      \end{itemize}
    \item<2-> R is an {\emph{object-oriented language}}.
    \item<3-> {\bf{Objects}} are like R's nouns: they are {\bf{things}}.
  \end{itemize}
}

\begin{frame}[fragile]
  \frametitle{The Basics: Objects (2)}
  {\LARGE{For example:}}

\begin{kframe}
\begin{alltt}
\hlcomment{# Add 2 + 2}

2 + 2
\end{alltt}
\end{kframe}
[1] 4

\begin{kframe}\begin{alltt}

\hlcomment{# Put the answer of 2 + 2 in an object called Answer}

Answer <- 2 + 2
\end{alltt}
\end{kframe}

\end{frame}

\frame{
  \frametitle{Assignemt}
  The {\tt{<-}} is the {\bf{assignment operator}} it assigns something to an object.
}

\frame{
  \frametitle{Tasks 1}
  Create {\bf{5}} different objects. Explore their properties. \\[0.5cm]
  
  What can you put into an object? \\[0.5cm]
  
  What could you not put into an object?
}

\begin{frame}[fragile]
  \frametitle{Basic Object Modes}
      {\LARGE{Basic Object Modes}} \\[0.5cm]
      All objects have a {\bf{mode}} (sometimes called data type or class). \\[0.5cm]
      Very basic object types are {\bf{numeric}}, {\bf{character strings}}, and {\bf{logical}} (TRUE or FALSE). \\[0.25cm]
      Use the \texttt{class} command to find out what an object's class is. For example:

\begin{knitrout}
\definecolor{shadecolor}{rgb}{0.969, 0.969, 0.969}\color{fgcolor}\begin{kframe}
\begin{alltt}
\hlfunctioncall{class}(Answer)
\end{alltt}
\begin{verbatim}
## [1] "numeric"
\end{verbatim}
\end{kframe}
\end{knitrout}

\end{frame}

\frame{
  \frametitle{Basic Object Types}
  {\LARGE{Basic Object Types}} \\[0.5cm]
  \begin{itemize}
    \item<1-> The main type of object we will use in this class is called a {\bf{data frame}}.
        \begin{itemize}
          \item<1-> We will cover data frames in detail next class. \\[0.5cm]
        \end{itemize}
      \item<2-> Today, lets look at some more basic R objects:
        \begin{itemize}
          \item<3-> Vectors
          \item<4-> Matrices
        \end{itemize}
    \end{itemize}
}

\frame{
  \frametitle{Vectors}
  {\LARGE{Vectors}} \\[0.5cm]
  Vectors are objects with multiple numbers {\emph{or}} character strings in a particular order. \\[0.5cm]
  They are the ``workhorse" of R (Matloff 2011).
}

\begin{frame}[fragile,plain]

\begin{knitrout}
\definecolor{shadecolor}{rgb}{0.969, 0.969, 0.969}\color{fgcolor}\begin{kframe}
\begin{alltt}
\hlcomment{# A vector of a sequence of numbers}
Sequence <- 1:10
Sequence
\end{alltt}
\begin{verbatim}
##  [1]  1  2  3  4  5  6  7  8  9 10
\end{verbatim}
\begin{alltt}

\hlcomment{# Non-sequntial numbers}
NonSeq <- \hlfunctioncall{c}(1, 30, 53)
NonSeq
\end{alltt}
\begin{verbatim}
## [1]  1 30 53
\end{verbatim}
\begin{alltt}

\hlcomment{# A character string vector}
CharVector <- \hlfunctioncall{c}(\hlstring{"Christopher"}, \hlstring{"John"}, \hlstring{"Gandrud"})
CharVector
\end{alltt}
\begin{verbatim}
## [1] "Christopher" "John"        "Gandrud"
\end{verbatim}
\end{kframe}
\end{knitrout}

\end{frame}

\frame{
  \frametitle{Combine Values Into a Vector}
    {\LARGE{\texttt{c}}} \\[0.5cm]
    The \texttt{c} command combines values into a vector.
}

\frame{
  \frametitle{Matrices 1}
  {\LARGE{Matrices}} \\[0.5cm]
  Matrices are like vectors, except they have {\bf{multiple columns}}. \\[0.5cm]
  You may remember matrices from math class:
  \[
 \begin{bmatrix}
  a & b & c \\
  d & e & f \\
  g & h & i
 \end{bmatrix}
\]
  {\bf{Don't worry}}, we never need to do matrix math in this class. R does it for us!
}

\frame{
  \frametitle{Matrices 2}
  We can use the \texttt{cbind} function (Column Bind) to combine two of the vectors we created before. 
}

\begin{frame}[fragile,plain]
\begin{knitrout}
\definecolor{shadecolor}{rgb}{0.969, 0.969, 0.969}\color{fgcolor}\begin{kframe}
\begin{alltt}
\hlcomment{# Bind the CharVector and NonSeq objects}
NewMatrix <- \hlfunctioncall{cbind}(CharVector, NonSeq)

\hlcomment{# Display contents of NewMatrix}
NewMatrix
\end{alltt}
\begin{verbatim}
##      CharVector    NonSeq
## [1,] "Christopher" "1"   
## [2,] "John"        "30"  
## [3,] "Gandrud"     "53"
\end{verbatim}
\end{kframe}
\end{knitrout}

\end{frame}

\frame{
  \frametitle{Tasks 3}
  Create a:
    \begin{itemize}
      \item Numeric vector
      \item Character vector
      \item Character matrix
      \item Numeric matrix
    \end{itemize}
}

\frame{
  \frametitle{Commands, Functions, Arguments (1)}
  {\LARGE{Commands \& Functions}} \\[0.5cm]
  Commands and Functions tell R to {\bf{do something}}. \\
  Usually they do something to an object.
  They are like R's {\bf{verbs}}.
}

\begin{frame}[fragile]
  \frametitle{Commands, Functions, Arguments (2)}
  {\LARGE{For example:}} \\[0.5cm]
  Lets create a set of 7 numbers: 1, 2, 3, 4, 6, 6, 7:

\begin{kframe}
\begin{alltt}
Numbers <- \hlfunctioncall{c}(1, 2, 3, 4, 6, 6, 7)
\end{alltt}
\end{kframe}


Now lets take the mean (average) of these 5 numbers with the {\tt{mean}} command

\begin{kframe}
\begin{alltt}
NumbersMean <- \hlfunctioncall{mean}(Numbers)

NumbersMean
\end{alltt}
\end{kframe}
[1] 4.143



\end{frame}

\frame{
  \frametitle{Commands, Functions, Arguments (1)}
  {\LARGE{Arguments}} \\[0.5cm]
  Arguments modify the command.
}

\begin{frame}[fragile]
  \frametitle{Commands, Functions, Arguments (2)}
  {\LARGE{For example:}} \\[0.5cm]
  Find what arguments the {\tt{round}} command can take by typing:
\begin{knitrout}
\definecolor{shadecolor}{rgb}{0.969, 0.969, 0.969}\color{fgcolor}\begin{kframe}
\begin{alltt}
?round
\end{alltt}
\end{kframe}
\end{knitrout}


  This gives us the {\bf{help file}} for the {\tt{round}} command. \\[0.25cm]

  We can see that one argument is {\tt{digits}} which specifies the number of decimal places to round to. \\ [0.25cm]

  To add the {\tt{digits}} argument just use the {\tt{=}} like this:

\begin{kframe}
\begin{alltt}
\hlfunctioncall{round}(NumbersMean, digits = 1)
\end{alltt}
\end{kframe}
[1] 4.1



\end{frame}

\begin{frame}[fragile]
  \frametitle{Nesting Commands}
  You can {\bf{nest}} commands inside of one another \\[0.5cm]
  This can sometimes make your code shorter.
\begin{knitrout}
\definecolor{shadecolor}{rgb}{0.969, 0.969, 0.969}\color{fgcolor}\begin{kframe}
\begin{alltt}
\hlcomment{# Find Numbers' mean and round it to one decimal place}
\hlfunctioncall{round}(\hlfunctioncall{mean}(Numbers), digits = 1)
\end{alltt}
\begin{verbatim}
## [1] 4.1
\end{verbatim}
\end{kframe}
\end{knitrout}

\end{frame}

\frame{
  \frametitle{Tasks 3}
  Find and use {\bf{2}} other commands. (Hint: \texttt{library(help = "base")} or \texttt{library(help = "graphics")}) \\[0.25cm]
  Explore their properties. \\[0.5cm]
}

\frame{
  \frametitle{Installing New Functions}
  {\LARGE{Installing New Functions}} \\[0.5cm]
  One of R's great strengths is that there are thousands of free, open source add-on packages that let you greatly expand what you can do in R. \\[0.25cm]
  To {\bf{install}} these packages use the \texttt{install.packages} command. \\[0.25cm]
  Once the package is installed you can {\bf{load it}} in your R session with the \texttt{library} function.
}

\frame{
  \frametitle{Tasks 4}
  Install and load the {\emph{WDI}} package.
}

\frame{
  \frametitle{Syllabus Change}
  Please check the online version of the syllabus. \\[0.5cm]
  I made a small change to the readings for the next two weeks.

}

\end{document}
