\documentclass{beamer}\usepackage{graphicx, color}
%% maxwidth is the original width if it is less than linewidth
%% otherwise use linewidth (to make sure the graphics do not exceed the margin)
\makeatletter
\def\maxwidth{ %
  \ifdim\Gin@nat@width>\linewidth
    \linewidth
  \else
    \Gin@nat@width
  \fi
}
\makeatother

\IfFileExists{upquote.sty}{\usepackage{upquote}}{}
\definecolor{fgcolor}{rgb}{0.2, 0.2, 0.2}
\newcommand{\hlnumber}[1]{\textcolor[rgb]{0,0,0}{#1}}%
\newcommand{\hlfunctioncall}[1]{\textcolor[rgb]{0.501960784313725,0,0.329411764705882}{\textbf{#1}}}%
\newcommand{\hlstring}[1]{\textcolor[rgb]{0.6,0.6,1}{#1}}%
\newcommand{\hlkeyword}[1]{\textcolor[rgb]{0,0,0}{\textbf{#1}}}%
\newcommand{\hlargument}[1]{\textcolor[rgb]{0.690196078431373,0.250980392156863,0.0196078431372549}{#1}}%
\newcommand{\hlcomment}[1]{\textcolor[rgb]{0.180392156862745,0.6,0.341176470588235}{#1}}%
\newcommand{\hlroxygencomment}[1]{\textcolor[rgb]{0.43921568627451,0.47843137254902,0.701960784313725}{#1}}%
\newcommand{\hlformalargs}[1]{\textcolor[rgb]{0.690196078431373,0.250980392156863,0.0196078431372549}{#1}}%
\newcommand{\hleqformalargs}[1]{\textcolor[rgb]{0.690196078431373,0.250980392156863,0.0196078431372549}{#1}}%
\newcommand{\hlassignement}[1]{\textcolor[rgb]{0,0,0}{\textbf{#1}}}%
\newcommand{\hlpackage}[1]{\textcolor[rgb]{0.588235294117647,0.709803921568627,0.145098039215686}{#1}}%
\newcommand{\hlslot}[1]{\textit{#1}}%
\newcommand{\hlsymbol}[1]{\textcolor[rgb]{0,0,0}{#1}}%
\newcommand{\hlprompt}[1]{\textcolor[rgb]{0.2,0.2,0.2}{#1}}%

\usepackage{framed}
\makeatletter
\newenvironment{kframe}{%
 \def\at@end@of@kframe{}%
 \ifinner\ifhmode%
  \def\at@end@of@kframe{\end{minipage}}%
  \begin{minipage}{\columnwidth}%
 \fi\fi%
 \def\FrameCommand##1{\hskip\@totalleftmargin \hskip-\fboxsep
 \colorbox{shadecolor}{##1}\hskip-\fboxsep
     % There is no \\@totalrightmargin, so:
     \hskip-\linewidth \hskip-\@totalleftmargin \hskip\columnwidth}%
 \MakeFramed {\advance\hsize-\width
   \@totalleftmargin\z@ \linewidth\hsize
   \@setminipage}}%
 {\par\unskip\endMakeFramed%
 \at@end@of@kframe}
\makeatother

\definecolor{shadecolor}{rgb}{.97, .97, .97}
\definecolor{messagecolor}{rgb}{0, 0, 0}
\definecolor{warningcolor}{rgb}{1, 0, 1}
\definecolor{errorcolor}{rgb}{1, 0, 0}
\newenvironment{knitrout}{}{} % an empty environment to be redefined in TeX

\usepackage{alltt}
\usetheme{Stats}
\setbeamercovered{transparent}
\usepackage{color}
\usepackage{hyperref}
  \hypersetup{
  	colorlinks=true
		linkcolor=black
		}
\usepackage{url}
\usepackage{graphics}
\usepackage{tikz}
\usepackage{booktabs}





%%%%%%%%%%%%%%%%%%%%%%%%%%%%%%%% Title Slide %%%%%%%%%%%%%%%%%%%%%%%%%%
\title[]{Intro to Social Science Data Analysis \\[1cm] Week 14: Statistical Analysis and Visualization of Results}
\author[]{
    \href{mailto:gandrud@yonsei.ac.kr}{Christopher Gandrud}
}
\date{\today}


\begin{document}

\frame{\titlepage}

\section[Outline]{}
\frame{\tableofcontents}

%%%%%%%%%%%% Schedule
\section{Schedule}
\frame{
  \frametitle{Schedule}
  We are dedicating \textbf{all} of the class time for the rest of the course to the research project. \\[0.5cm]
  Schedule:
  \begin{itemize}
    \item<2-> Week 13: Research question, design, \& data download,
    \item<3-> Week 14: Statistical Analysis \& Results Visualization,
    \item<4-> Week 15: Write up.
    \item<5-> Week 16: Presentations.
  \end{itemize}
}

%%%%%%%%%% Goals
\section{This Week's Goals}
\frame{
	\frametitle{Today's Goals}
	{\LARGE{Today's Goals:}} \\[0.5cm]
	\begin{enumerate}
		\item Make sure your data is clean \& ready for analysis.
		\item Descriptive statistics/begin to thinik about linear model assumptions.
	\end{enumerate}
}

\frame{
  \frametitle{Thursday's Goals}
{\LARGE{Thursday's Goals:}} \\[0.5cm]
	\begin{itemize}
		\item Run your inferential statistics.
	\end{itemize}  
}

%%%%%%%%%%%% Robust
\section{Robust Standard Errors for Dependent Data}
\frame{
  \frametitle{Dependent Data}
{\LARGE{Remember:}} \\[0.5cm]
  One of the assumptions of linear regression is that the observations are independent of one another. \\[0.5cm]
  Many of you are using data from many countries across many years. \\[0.5cm]
  This is called \textbf{time-series cross-sectional data.} \\[0.5cm]
  This type of data often has \textbf{biased standard errors}.
}

\begin{frame}[fragile]
  \frametitle{For Example}
  {\LARGE{Example Time-series Cross Sectional Data}}
\begin{knitrout}
\definecolor{shadecolor}{rgb}{0.969, 0.969, 0.969}\color{fgcolor}\begin{kframe}
\begin{verbatim}
##        country year GDPperCapita InfantMort
## 29 Afghanistan 2002        158.0       90.5
## 30 Afghanistan 2003        168.7       88.4
## 31 Afghanistan 2004        196.2       86.4
## 32 Afghanistan 2005        227.9       84.3
## 37     Albania 2002       1440.0       20.9
## 38     Albania 2003       1819.4       19.8
\end{verbatim}
\end{kframe}
\end{knitrout}

\end{frame}

\begin{frame}[fragile]
  \frametitle{One Solution}
  A common way of handling data like this is to use \textbf{robust standard errors}. \\[0.5cm]
  They are easy to implement with \emph{Zelig}


\begin{knitrout}
\definecolor{shadecolor}{rgb}{0.969, 0.969, 0.969}\color{fgcolor}\begin{kframe}
\begin{alltt}
M1 <- \hlfunctioncall{zelig}(InfantMort ~ GDPperCapita, 
            model = \hlstring{"normal"},
            data = MortalityGDP,
            robust = TRUE,
            cite = FALSE)
\end{alltt}
\end{kframe}
\end{knitrout}

\end{frame}

\frame{
  \frametitle{Robust Standard Error Note}
{\LARGE{Note:}} \\[0.5cm]
  It's (usually) a good thing if the robust and regular standard errors are basically the same. \\[0.5cm]
  This indicates you are not violating the model assumption. \\[0.5cm]
  For a much more advanced discussion see: \url{http://gking.harvard.edu/files/robust.pdf} \\[0.5cm]
  There is much more to learn on this topic, which we won't cover in this class.
}

%%%%%%%%%% xtable
\section{Results tables with xtable}
\frame{
  \frametitle{Results Tables}
  Hand typing results tables is really irritating. \\[0.5cm]
  You can use the \emph{xtable} package to automate table creation. \\[0.5cm]
  See: \url{http://bit.ly/TFxDS4}.
}


\end{document}
